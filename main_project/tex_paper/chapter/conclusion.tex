%Conclusion
As to be expected the prediction performance of the machine learning algorithms was better when trained on the larger datasets.
Also a great difference in the number of binding vs non-binding peptides generally degraded the performance.
The trend for the larger datasets was the 10 hidden-layer non-blosum matrix encoded data was performing best. 
This could demonstrate that the large datasets are covering more the space of binding/non-binding data and training without using blosum matrix encoding is more accurate.
It is a little difficult to see from the graphs but for the small datasets, using the blosum matrix increased the predictive performance.
The number of hidden neurons also affected the performance. We tried 1, 2 and 10 hidden neurons, the 10 performing best in most cases.

SVM had worse Pearsons correlation coefficient in all cases compared to the ANN's. Also using a polynomial kernel function of $1_{st}$ degree was better than of $2_{nd}$ degree.

The PSSM method was straight forward and performed remarkably well, alsmost the same as SVM for most cases.
