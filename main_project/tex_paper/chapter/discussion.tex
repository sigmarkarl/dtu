% Discussion
Although quite successful the results of the machine learning methods do not give one much insight into the 'mechanical' workings of the MHC binding.
As a pattern recognition tool, ANN with peptide binding data could give some clues for further study on the crystal structury of the MHC proteins.
As stated in the class exercise slides, the ANN's seem to have had a bad reputation. 
It is our guess that the good performance of ANN's in this case is because they are well suited to solve the MHC peptide binding problem. 
Also the correct use and training of the ANN's in terms of overfitting is an helping factor.
