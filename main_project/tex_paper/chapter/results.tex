% Results

C programs from the course were used to estimate MHC binding affinity with PSSM. 
Here a threshold must be used to limit included peptides to the ones that actually bind to the MHC molecule. 
In the machine learing methods non-binding data is also be valuable in the estimation.

All entries from the training data set with binding affinity over 0.426 are feeded into a program. 
This value means that the peptide successfully binds to the MHC molecule. All peptides should be of same length.
The Blosum frequency substitution matrix, is a
conditional probability matrix of matching amino
acids j given you have amino acid i

\begin{equation}
P(j|i) = \frac{P_ij}{Q_i}
\end{equation}

The matrix is used to. The result is a PSSM matrix. 

Experimental data from the HLA-A*0201 allele is randomly split up in training and evaluation sections, the evaluation section being 1/5 of the training data size (618 peptides vs 2471 peptides in the training set).
For 5 different samplings we get a Pearson correlation coefficients of {0.75323,, 0.78415, 0.78009, 0.78896, 0.76480}, the average being 0.774.
The corresponding results without using sequence weighting are {0.73530, 0.75961, 0.75791, 0.76343, 0.74639}, with average of 0.753
Using identity matrix instead of the blosum substitiution matrix gives similar values {0.75107, 0.77989, 0.74344, 0.78552, 0.76072} with average of 0.764.
The blosum matrix is not making a difference here as substitution frequencies should not affect much binding affinity of a peptide in a MHC molecule.

We also try the smaller set HLA-A3001. It has a total of 669 experimentally verified peptides.
Using the blosum matrix the correlation results are {0.69872, 0.55787, 0.65032, 0.55827, 0.57044}, an average of 0.607
The corresponding results without using sequence weighting are {0.67775, 0.54853, 0.63731, 0.55026, 0.56416}, with average of 0.596
The identity matrix gives results of with average of (0.63987 0.45267 0.55898 0.49296 0.49224) 0.527 

Using a small dataset the benefit of using a blosum frequency substitution matrix is greater.



