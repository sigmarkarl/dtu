% Results



Although quite successful the results of the machine learning methods do not give one much insight into the 'mechanical' workings of the MHC binding.
PSSM weight matrices can also quite accurately describe a sequence motif like MHC class I. In fact their results come quite close to the machine learning methods for the larger datasets. 
Based on results from the three alleles A*0201, A*3001 and B*4001, the ANN's are though performing best overall.
As a pattern recognition tool, ANN with peptide binding data could give some clues for further study on the crystal structury of the MHC proteins.
As stated in the class exercise slides, the ANN's seem to have had a bad reputation. 
It is our guess that the good performance of ANN's in this case is because they are well suited to solve the MHC peptide binding problem. 
Also the correct use and training of the ANN's in terms of overfitting is an helping factor.
By the way in this section we will list the results obtained testing the PSSM, SVM and ANN algorithms on 35 HLA alleles.

\subsection*{PSSM}


C programs from the course were used to estimate MHC binding affinity with PSSM. 
Here a threshold must be used to limit included peptides to the ones that actually bind to the MHC molecule. 
In the machine learning methods non-binding data is also be valuable in the estimation.

All entries from the training data set with binding affinity over 0.426 are feeded into a program. 
This value means that the peptide successfully binds to the MHC molecule. All peptides have the same length.

To prevent overfitting in the creation of the PSSM matrix, we split the data into 5 parts. 4/5 are used in creating the matrix and 1/5 in evaluating it.
Experimental data from the HLA-A*0201 allele is randomly split up in training and evaluation sections, the evaluation section being 1/5 of the training data size (618 peptides vs 2471 peptides in the training set).
For 5 different samplings and using sequence weighting we get a Pearson correlation coefficients of {0.75323,, 0.78415, 0.78009, 0.78896, 0.76480}, the average being 0.774 (see Tab \ref{tab:pssm1} first column).
We also try the smaller set HLA-A*3001. It has a total of 669 experimentally verified peptides.
Using sequence weighting the correlation results are 0.69872, 0.55787, 0.65032, 0.55827 and 0.57044 (see Tab \ref{tab:pssm1} Second Column), an average of 0.607
The corresponding results without using sequence weighting are 0.67775, 0.54853, 0.63731, 0.55026 and 0.56416, with average of 0.596.
From Figure \ref{fig:pssm1} it interesting the case of HLA-B*4001, in which even if there are a considerable number of peptides, the algorithm have poor results (Tab \ref{tab:pssm1} third column). Looking at Table \ref{ftable} is it clear that the little percentage of binding peptides is the cause of so poor performance. 
As seen in Figure \ref{fig:pssm1}, using sequence weighting is gives better results in most cases (of 35 alleles).

Using a small dataset the benefit of using a blosum frequency substitution matrix is greater.
\begin{table*}\scriptsize

\begin{center}

\begin{tabular}{rrr}

\begin{tabular}{rllrrr}
  \hline
 & Param & Allele & Sample & Size & PCC \\ 
  \hline
 & PSSM w/o SW & A0201 &   0 & 618 & 0.74 \\ 
 & PSSM w/o SW & A0201 &   1 & 618 & 0.76 \\ 
 & PSSM w/o SW & A0201 &   2 & 618 & 0.76 \\ 
 & PSSM w/o SW & A0201 &   3 & 618 & 0.76 \\ 
 & PSSM w/o SW & A0201 &   4 & 617 & 0.75 \\ 
\hline
 & PSSM w SW & A0201 &   0 & 618 & 0.75 \\ 
 & PSSM w SW & A0201 &   1 & 618 & 0.78 \\ 
 & PSSM w SW & A0201 &   2 & 618 & 0.78 \\ 
 & PSSM w SW & A0201 &   3 & 618 & 0.79 \\ 
 & PSSM w SW & A0201 &   4 & 617 & 0.76 \\ 
   \hline
\end{tabular}

\begin{tabular}{rllrrr}
  \hline
 & Param & Allele & Sample & Size & PCC \\ 
  \hline
   & PSSM & A3001 &   0 & 134 & 0.68 \\ 
   & PSSM & A3001 &   1 & 134 & 0.55 \\ 
   & PSSM & A3001 &   2 & 134 & 0.64 \\ 
   & PSSM & A3001 &   3 & 134 & 0.55 \\ 
   & PSSM & A3001 &   4 & 133 & 0.56 \\ 
  \hline
   & PSSM SW & A3001 &   0 & 134 & 0.70 \\ 
   & PSSM SW & A3001 &   1 & 134 & 0.56 \\ 
   & PSSM SW & A3001 &   2 & 134 & 0.65 \\ 
   & PSSM SW & A3001 &   3 & 134 & 0.56 \\ 
   & PSSM SW & A3001 &   4 & 133 & 0.57 \\ 
   \hline
\end{tabular}

\begin{tabular}{rllrrr}
  \hline
 & Param & Allele & Sample & Size & PCC \\ 
  \hline
   & PSSM & B4001 &   0 & 216 & 0.19 \\ 
   & PSSM & B4001 &   1 & 216 & 0.24 \\ 
   & PSSM & B4001 &   2 & 216 & 0.19 \\ 
   & PSSM & B4001 &   3 & 215 & 0.38 \\ 
   & PSSM & B4001 &   4 & 215 & 0.29 \\ 
\hline
   & PSSM SW & B4001 &   0 & 216 & 0.23 \\ 
   & PSSM SW & B4001 &   1 & 216 & 0.29 \\ 
   & PSSM SW & B4001 &   2 & 216 & 0.24 \\ 
   & PSSM SW & B4001 &   3 & 215 & 0.41 \\ 
   & PSSM SW & B4001 &   4 & 215 & 0.32 \\ 
   \hline
\end{tabular}

\end{tabular}
\end{center}
\caption{Summary of the PSSM results for three most significant alleles, A0201, A3001 and B4001. Looking also at the data in Tab \ref{ftable} we can see that even if B4001 have a grat number of peptides in the dataset, only the 4\% is binding. This is translate in the results with a really bad estimation of the binding}\label{tab:pssm1}
\end{table*}

\subsection*{SVM}
In this investigation version 3.6.3 of the java based Weka sofware is used for SVM calculations.
The data has to be prepared for input into the Weka program.
The number of amino acids is 20 and the length of the peptide is 9 so each peptide is changed to a vector of length 20*9. 
Three encoding schemes are used, blosum50, sparse and z-score. 
In the case of the blosum encoding scheme, each of the 9 sections of length 20 simply contain the data from each row in blosum matrix corresponding to the amino acid in question.
The sparse encoding uses the identity substitution matrix. 
The z-score encoding is a condensed manner to represent each amino acid, only 5 values are used to encode each amino acid based on on their properties. 
These properties were deduced from measured data using thin-layer chromatography and nuclear magnetic resonance and some calculated variables 
such as side chain charge, hydrogren bond donor and acceptor properties, log P and molecular weight. 
Each row in the input matrix has therefore length of 5*20+1 (the one being the measured affinity value)
The same data as from the PSSM section is used for comparison. For the HLA-A*0201 dataset and sparse encoding we get the results displayed in the first line of table \ref{tab:svm1}.

For the A*0201 allele, the SMO classifier (Table \ref{tab:svm1} second line) using sparse encoded data with polynomial kernel of first degree gives slightly better results than the PSSM. Raising the degree of the kernel to 2 does not improve the results.

Using the blosum matrix with a kernel of degree of 1 gives almost the same results as a sparse encoding (Table \ref{tab:svm1} the 3$^{rd}$ line). Polynomial kernel of degree 2 has considerably worse performance for the blosum encoded data (see Tab \ref{tab:svm1} line 4 ). Also the z-score encoded data with polynomial kernel of degree 1 (Tab \ref{tab:svm1} line 5) does not give better results.
Trying the same tests with a smaller dataset, the HLA-A*3001 (Tab \ref{tab:svm1} in the 6$^{th}$ line), we notice that blosum encoded data slightly raises the prediction performance (Tab \ref{tab:svm1} in the 7$^{th}$ line).

\begin{table}[ht]\scriptsize
\begin{center}
\begin{tabular}{rllrrr}
  \hline
 & Param & Allele & Size & PCC & MAE \\ 
  \hline
 & SVM Pol. 1$^{st}$ d + Sparse & A0201 &   618 & 0.78 & 0.1524 \\ 
 & SVM Pol. 2$^{st}$ degree & A0201 &   618 & 0.756 & 0.1535 \\ 
 & SVM Pol. 1$^{st}$ d + Blosum & A0201 &   618 & 0.7789 & 0.1533 \\ 
 & SVM Pol. 2$^{nd}$ degree & A0201 &   618 & 0.6692 & 0.2047 \\ 
 & SVM Pol. 1$^{st}$ d + z-score & A0201 &   618 & 0.6888 & 0.1802 \\ 
 & SVM Sparse & A3001 &   134 & 0.7412 & 0.1008 \\ 
 & SVM Pol. 1$^{nd}$ d + Blosum & A3001 &   134 & 0.7671 & 0.0945 \\ 
 & SVM Pol. 1$^{nd}$ d + Sparse & B4001 &   216 & 0.4876 & 0.0373 \\ 
 & SVM Pol. 1$^{nd}$ d + Blosum & B4001 &   216 & 0.4456 & 0.0387 \\ 
 & SVM Pol. 1$^{nd}$ d + Zscore & B4001 &   216 & 0.2397 & 0.0400 \\ 
   \hline
\end{tabular}
\end{center}
\caption{Result table for SVM with various kernel and various encoding schema. The three alleles A0201, A3001 and B4001 were cosidered.}\label{tab:svm1}
\end{table}

\begin{table*}[hb]\scriptsize
\begin{center}
\begin{tabular}{rrr}

\begin{tabular}{rllrrr}
  \hline
 & Param & Allele & Sample & Size & PCC \\ 
  \hline
   & NN 10HL & A0201 &   0 & 618 & 0.84 \\ 
   & NN 10HL & A0201 &   1 & 618 & 0.87 \\ 
   & NN 10HL & A0201 &   2 & 618 & 0.86 \\ 
   & NN 10HL & A0201 &   3 & 618 & 0.86 \\ 
   & NN 10HL & A0201 &   4 & 617 & 0.87 \\ 
   \hline
   & NN 10HL Blosum & A0201 &   0 & 618 & 0.83 \\ 
   & NN 10HL Blosum & A0201 &   1 & 618 & 0.86 \\ 
   & NN 10HL Blosum & A0201 &   2 & 618 & 0.85 \\ 
   & NN 10HL Blosum & A0201 &   3 & 618 & 0.86 \\ 
   & NN 10HL Blosum & A0201 &   4 & 617 & 0.86 \\ 
   \hline
   & NN 1HL & A0201 &   0 & 618 & 0.84 \\ 
   & NN 1HL & A0201 &   1 & 618 & 0.87 \\ 
   & NN 1HL & A0201 &   2 & 618 & 0.86 \\ 
   & NN 1HL & A0201 &   3 & 618 & 0.86 \\ 
   & NN 1HL & A0201 &   4 & 617 & 0.87 \\ 
   \hline
   & NN 1HL Blosum & A0201 &   0 & 618 & 0.84 \\ 
   & NN 1HL Blosum & A0201 &   1 & 618 & 0.87 \\ 
   & NN 1HL Blosum & A0201 &   2 & 618 & 0.86 \\ 
   & NN 1HL Blosum & A0201 &   3 & 618 & 0.86 \\ 
   & NN 1HL Blosum & A0201 &   4 & 617 & 0.87 \\ 
   \hline
   & NN 2HL & A0201 &   0 & 618 & 0.84 \\ 
   & NN 2HL & A0201 &   1 & 618 & 0.87 \\ 
   & NN 2HL & A0201 &   2 & 618 & 0.86 \\ 
   & NN 2HL & A0201 &   3 & 618 & 0.86 \\ 
   & NN 2HL & A0201 &   4 & 617 & 0.87 \\ 
   \hline
   & NN 2HL Blosum & A0201 &   0 & 618 & 0.85 \\ 
   & NN 2HL Blosum & A0201 &   1 & 618 & 0.87 \\ 
   & NN 2HL Blosum & A0201 &   2 & 618 & 0.85 \\ 
   & NN 2HL Blosum & A0201 &   3 & 618 & 0.87 \\ 
   & NN 2HL Blosum & A0201 &   4 & 617 & 0.87 \\ 
   \hline
\end{tabular}


\begin{tabular}{rllrrr}
  \hline
 & Param & Allele & Sample & Size & PCC \\ 
  \hline
   & NN 10HL & A3001 &   0 & 134 & 0.81 \\ 
   & NN 10HL & A3001 &   1 & 134 & 0.66 \\ 
   & NN 10HL & A3001 &   2 & 134 & 0.80 \\ 
   & NN 10HL & A3001 &   3 & 134 & 0.67 \\ 
   & NN 10HL & A3001 &   4 & 133 & 0.71 \\ 
\hline
   & NN 10HL Blosum & A3001 &   0 & 134 & 0.82 \\ 
   & NN 10HL Blosum & A3001 &   1 & 134 & 0.65 \\ 
   & NN 10HL Blosum & A3001 &   2 & 134 & 0.78 \\ 
   & NN 10HL Blosum & A3001 &   3 & 134 & 0.68 \\ 
   & NN 10HL Blosum & A3001 &   4 & 133 & 0.62 \\ 
\hline
   & NN 1HL & A3001 &   0 & 134 & 0.83 \\ 
   & NN 1HL & A3001 &   1 & 134 & 0.66 \\ 
   & NN 1HL & A3001 &   2 & 134 & 0.81 \\ 
   & NN 1HL & A3001 &   3 & 134 & 0.67 \\ 
   & NN 1HL & A3001 &   4 & 133 & 0.72 \\ 
\hline
   & NN 1HL Blosum & A3001 &   0 & 134 & 0.84 \\ 
   & NN 1HL Blosum & A3001 &   1 & 134 & 0.66 \\ 
   & NN 1HL Blosum & A3001 &   2 & 134 & 0.80 \\ 
   & NN 1HL Blosum & A3001 &   3 & 134 & 0.67 \\ 
   & NN 1HL Blosum & A3001 &   4 & 133 & 0.59 \\ 
\hline
   & NN 2HL & A3001 &   0 & 134 & 0.82 \\ 
   & NN 2HL & A3001 &   1 & 134 & 0.66 \\ 
   & NN 2HL & A3001 &   2 & 134 & 0.81 \\ 
   & NN 2HL & A3001 &   3 & 134 & 0.67 \\ 
   & NN 2HL & A3001 &   4 & 133 & 0.71 \\ 
\hline
   & NN 2HL Blosum & A3001 &   0 & 134 & 0.84 \\ 
   & NN 2HL Blosum & A3001 &   1 & 134 & 0.66 \\ 
   & NN 2HL Blosum & A3001 &   2 & 134 & 0.80 \\ 
   & NN 2HL Blosum & A3001 &   3 & 134 & 0.68 \\ 
   & NN 2HL Blosum & A3001 &   4 & 133 & 0.60 \\ 
   \hline
\end{tabular}

\begin{tabular}{rllrrr}
  \hline
 & Param & Allele & Sample & Size & PCC \\ 
  \hline
   & NN 10HL & B4001 &   0 & 216 & 0.58 \\ 
   & NN 10HL & B4001 &   1 & 216 & 0.65 \\ 
   & NN 10HL & B4001 &   2 & 216 & 0.60 \\ 
   & NN 10HL & B4001 &   3 & 215 & 0.59 \\ 
   & NN 10HL & B4001 &   4 & 215 & 0.65 \\ 
\hline
   & NN 10HL Blosum & B4001 &   0 & 216 & 0.61 \\ 
   & NN 10HL Blosum & B4001 &   1 & 216 & 0.62 \\ 
   & NN 10HL Blosum & B4001 &   2 & 216 & 0.56 \\ 
   & NN 10HL Blosum & B4001 &   3 & 215 & 0.60 \\ 
   & NN 10HL Blosum & B4001 &   4 & 215 & 0.60 \\ 
\hline
   & NN 1HL & B4001 &   0 & 216 & 0.59 \\ 
   & NN 1HL & B4001 &   1 & 216 & 0.67 \\ 
   & NN 1HL & B4001 &   2 & 216 & 0.64 \\ 
   & NN 1HL & B4001 &   3 & 215 & 0.63 \\ 
   & NN 1HL & B4001 &   4 & 215 & 0.68 \\ 
\hline
   & NN 1HL Blosum & B4001 &   0 & 216 & 0.64 \\ 
   & NN 1HL Blosum & B4001 &   1 & 216 & 0.59 \\ 
   & NN 1HL Blosum & B4001 &   2 & 216 & 0.55 \\ 
   & NN 1HL Blosum & B4001 &   3 & 215 & 0.63 \\ 
   & NN 1HL Blosum & B4001 &   4 & 215 & 0.58 \\ 
\hline
   & NN 2HL & B4001 &   0 & 216 & 0.60 \\ 
   & NN 2HL & B4001 &   1 & 216 & 0.66 \\ 
   & NN 2HL & B4001 &   2 & 216 & 0.63 \\ 
   & NN 2HL & B4001 &   3 & 215 & 0.62 \\ 
   & NN 2HL & B4001 &   4 & 215 & 0.67 \\ 
\hline
   & NN 2HL Blosum & B4001 &   0 & 216 & 0.64 \\ 
   & NN 2HL Blosum & B4001 &   1 & 216 & 0.60 \\ 
   & NN 2HL Blosum & B4001 &   2 & 216 & 0.56 \\ 
   & NN 2HL Blosum & B4001 &   3 & 215 & 0.63 \\ 
   & NN 2HL Blosum & B4001 &   4 & 215 & 0.59 \\ 
   \hline
\end{tabular}
\end{tabular}
\end{center}

\caption{Summary of the Artificial Neural Network algorithm results for three alleles: A0201, A3001 and B4001. Refearing to the data in Tab \ref{ftable} we can see that even if B4001 have a grat number of peptides in the dataset, only the 4\% is binding. By the way the performance of the ANN algorithm it is still better compared to the other algorithms in this article.}\label{tab:nn1}

\end{table*}

\subsection*{ANN}



The \textit{nnbackprop} and the \textit{nnforward} algorithms from the course were used in order to estimate the MHC binding affinity.
In the figure \ref{fig:ann1} and figure \ref{fig:ann2} are summarized the results for all the 35 alleles, ordered by the number of peptides in each sample.
The back propagation algorithm was tested with 10, 2 and 1 hidden layers, on data with and without Blosum encoding. The total combination of the given options amount then to 6 different configuration for each allele.
In table \ref{tab:nn1} are listed the results for each separate test for the three alleles A0201, A3001 and B4001. The average PCC for A0201 is 0.86, A3001 0.72 and B4001 had PCC average of 0.62.






