\documentclass[presentation]{beamer}   % to compile the presentation
%\documentclass[handout]{beamer}        % to compile 2x2 handouts
\usepackage[ansinew]{inputenc}
\usepackage[T1]{fontenc}
\usepackage{lmodern,textcomp}
\usepackage{breakurl}
\usepackage{graphicx}

\usetheme{Algo}
%\usetheme{Warsaw}

\usepackage[formats]{listings}
\lstdefineformat{C}{% 
	\{=\newline\string\newline\indent,% 
	\}=[;]\newline\noindent\string\newline,% 
	\};=\newline\noindent\string\newline,% 
	;=[\ ]\string\space}
\lstset{language=C}

%\usepackage{/usr/lib64/R/share/texmf/Sweave}
%\usepackage{/Library/Frameworks/R.framework/Resources/share/texmf/Sweave}
\setbeamercovered{transparent}

\begin{document}
\pgfdeclareimage[height=1cm]{Biglogo}{dtu_logo}
%\pgfdeclareimage[width=12mm]{footlogo}{partek_logo_s}

\author{Sigmar Stef\`{a}nson and Francesco Favero}
\title[PSSM,ANN,SVM]{Comparison of MHC peptide binding data classification using PSSM, SVM and ANN}
\date{15 Dic. 2010}
\titlegraphic{\pgfuseimage{Biglogo}} % Graphics for title slide
%\logo{\pgfuseimage{footlogo}} % The left logo



\begin{frame}
  \maketitle
\end{frame}

\begin{frame}
  \frametitle{Outline}
  \tableofcontents[currentsection]
\end{frame}

\section{Lists}
\begin{frame}
  \frametitle{A slide with a list}
  \framesubtitle{Use <number> to specify layers}
  \begin{itemize}
    \item<1-> First item on all overlays.
    \item<2>  Second item only on the second overlay.
    \item<1,3> Third on first and last.
      \begin{itemize}
        \item<3>  Lists can be nested.
        \item<3> Use this to structure your presentation.
      \end{itemize}
  \end{itemize}
\end{frame}

\section{Columns}
\begin{frame}
  \frametitle{A slide with columns}
  \begin{columns}[t] % Align the columns at the top
    \column{0.4\textwidth}
      This is the \alert{first} column. It occupies $40$\% of the text width.
    \column{0.6\textwidth}
      This is the \alert{second} column. This could be a nice image\ldots
      \begin{center}
        \rule{0.4\textwidth}{0.3\textwidth}
      \end{center}
  \end{columns}
\end{frame}

\section{Blocks}
\begin{frame}
  \frametitle{Use blocks to highlight your points}
  \begin{block}{<The title of the block>}
    This is the point you want to highlight. It could be an important formula
    \[
      a^2+b^2=c^2
    \]
  \end{block}

  \begin{example}
    The example block is useful for typesetting examples consistently.
  \end{example}
\end{frame}

\section{Verbatim}
\begin{frame}[fragile]
  \frametitle{Verbatim material}
  If the slide contains verbatim material you must use the \texttt{fragile} option for the frame.

  \begin{verbatim}
  This is verbatim text
  !"#�%&/()=?
  \end{verbatim}
  
  The \texttt{listings} package can be used for more fancy verbatim text and pretty printing of source code.
\end{frame}

\section{Finding documentation}
\begin{frame}
  \frametitle{Beamer documentation}
  The Beamer userguide is available on-line at CTAN:
  \begin{center}
    \url{ftp://tug.ctan.org/pub/tex-archive/macros/latex/contrib/beamer/doc/beameruserguide.pdf}
  \end{center}

  If Beamer is installed on your system you can find the manual by running
  \begin{center}
    \texttt{mthelp beamer}
  \end{center}
  in a Command Prompt.
\end{frame}

\section{DTU stuff}

\begin{frame}
  \frametitle{The template}
  This presentation template is a \texttt{beamer} implementation of the official DTU PowerPoint template available at
  \begin{center}
    \url{http://portalen.dtu.dk/Services/Kommunikation.aspx}
  \end{center}
  
  To follow the design guidelines completely, you should use the colors from the DTU color palette
  \begin{center}
    \url{http://portalen.dtu.dk/upload/ak/design/dtu-farvemanual_03_07_2006.pdf}
  \end{center}
  These colors are defined in the DTU beamer theme with the names shown on the next slide.
\end{frame}

\end{document} 
